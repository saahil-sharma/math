\documentclass{report}

\input{preamble}
\input{macros}
\input{letterfonts}
\title{\Huge{Topology Packet}\\Alfonso Gracia - Saz}
\author{\huge{Saahil Sharma}}
\usepackage{tikz}
\date{}
\begin{document}

\maketitle
\newpage
\pdfbookmark[section]{\contentsname}{toc}
\tableofcontents
\pagebreak

\chapter{}
\section{Exercise 1.2}
\begin{center} 
Among the following, some are topologies on the set $\mathbb{Z}$ and some are not. Which ones are? If an example is not a topology, but you can modify it slightly to make it into a topology, do so. If an example is a topology, and you can generalize it into more examples, do so.  
\end{center}
\qs{Exercise 1.2.a}{\[\tau = \{V \subseteq \mathbb{Z} \mid 0 \in V \} \text{ In other words, a set is open iff it contains } 0.\} \]}
\sol This is not a topology on $\mathbb{Z}$ because $\tau$ must contain $\varphi$, and $\varphi$ does not contain zero.

\qs{Exercise 1.2.b}{\[\tau = \{V \subseteq \mathbb{Z} \mid 0 \notin V \} \text{ In other words, a set is open iff it does not contain } 0.\} \]}
\sol This is not a valid topology on $X$, because if $0$ is not included, the total set will not be an open set, therefore $\tau$ is not valid.  

\qs{Exercise 1.2.c}{\[\tau = \{V \subseteq \mathbb{Z} \mid 0 \in V \} \text{ or } 1 \in V\} \]}.
\sol This is a topology on $\mathbb{Z}$ because it contains the empty set and the total set, and a union or intersection of $2$ open sets will be also be considered an open set. 

\qs{Exercise 1.2.d}{\[\tau = \{V \subseteq \mathbb{Z} \mid 0 \in V \} \text{ and } 1 \in V\} \]}.
\sol This is a topology on $\mathbb{Z}$ because the empty set will be included because the elements $0$ and $1$ do not have to be in every open set.

\qs{Exercise 1.2.e}{\[\tau = \{V \subseteq \mathbb{Z} \mid V \text{ is finite }\]}
\sol This is not a topology on $\mathbb{Z}$ as it requires that all sets inside must be finite and the total set is infinite.                                                         

\qs{Exercise 1.2.f}{\[\tau = \{V \subseteq \mathbb{Z} \mid V \text{ is infinite }\]}
\sol The total set is an infinite set, therefore this topology is invalid. 

\section{Exercise 1.3}
\begin{center}
    Among the following, which ones are topologies on the set $\mathbb{R}$ and which ones are not?
\end{center}
\qs{1.3.a}{\[\tau = \{(a, \infty) \mid a \in \mathbb{R}\} \cup \{\phi, \mathbb{R}\}\]}
\sol We can prove this topology is not valid by a proof of contradiction. Consider the claim to be true. Then, the union of $2$ open sets must also be an open set. Consider the $2$ sets where the first set has a starting value of $a$ and the second set has a starting value of $b$. Consider $a < b$. In this situation, the union of these $2$ sets will contain all values from $a$ to $\infty$, not inclusive. This set will not include $b$ though, therefore there is a missing value between $a$ and $\infty$ and the union of these two sets cannot be an open sets. This satisfies that (a) cannot be a valid topology. 

\qs{1.3.b}{\[\tau = \{[a, \infty) \mid a \in \mathbb{R}\} \cup \{\phi, \mathbb{R}\}\].} 
\sol We can prove this topology is valid by satisfying all $3$ axioms that define a topological space. The first axiom is automatically satisfied as the set is in union with the total set and the empty set. Continuing on, the union of $2$ sets must also be an empty set. This can be proved by considering $2$ sets, one with an $a$ value of $a$ and another with an $a$ value of $b$. If $a = b$, then their union is itself and the resulting set is open. Continuing on, if $a < b$, then the set will include all values from $a$ to $\infty$, including $b$, therefore this set would also be an open set. The same logic can be applied to the situation in which $b < a$, and for the $3$rd axiom, regarding intersections of $2$ sets. 

\section{Exercise 1.4}
\begin{center}
Let $X$ be any set. 
\end{center}
\qs{1.4.a}{What is the topology on $X$ that has the most open sets? This is called the \textit{discrete} topology on $X$.} 
\sol The topology with the most amount of open sets on $X$ will include the total set, the empty set, and every subset, and the power set $\mathbb{X}$ that includes all the subsets of $X$. 
\qs{1.4.b}{What is the topology on $X$ that has the least open sets? This is called the \textit{indiscrete} topology on $X$.}
\sol The topology with the least amount of open sets would only contain the total set, the empty set, and the union of those $2$ sets.  
\newpage
\section{Exercise 1.5}
\begin{center}
Let $X$ be an arbitrary set. Which ones of the following are topologies?
\end{center}

\qs{1.5.a}{The \textit{cofinite} topology: A set \[V \subseteq X \text{ is open iff } [ X \backslash  V \text{ is finite or } V = \phi]\].}
\sol We must prove this topology $\tau$ is in order with the $3$ defining points of a topology on $X$. 
Now, if we have two sets in $\tau$ being $V_i$ and $V_j$, then $X - (V_i \cup V_j)$ must also be infinite. This is from the second axiom of a topology. From De Morgan's Laws, we know that
                    \[
                        X - ( V_i \cup V_j ) = (X - V_i) \cap (X - V_j).                  
                    \]
This indicates the intersection of $2$ infinite sets, which can be either infinite, finite or empty. Therefore, there exists a counterexample and the \textit{coinfinite} topology does not exist for all sets $X$. 

\qs{1.5.c}{The \textit{cocountable} topology: A set $V \subseteq X$ is open iff \[ [X \backslash V \text{ is countable or  } V = \phi]\].} 
\sol To prove this, we must prove that the topology $\tau$ exists only for countable sets. That is, we have already proved when $X$ is finite in the first case. Let $K_i$ be equal to $X \backslash V_i$. Therefore, $\tau$ is equal to the collection of sets $V_i$, $V_j$, $\dots$. Therefore, we must prove that the union of $V_i$ and $V_j$ is also an open set. To do this, we use De Morgan's laws.
                    \[X - (V_i \cup V_j) = (X-V_i) \cap (X-V_j) \Longrightarrow  \]
                    \[X - (V_i \cup V_j) = (K_i) \cap (K_j) \].
By definition, both $K_i$ and $K_j$ are countable, therefore their union is countable. The same logic and be applied to the $3$rd axiom of a topology, proving that $\tau$ is a valid topology. A very nice question!                            

\section{Exercise 1.6}
\begin{center}
    Let $(X, \tau)$ be a topological space. Prove each of the following statements true or false.         
\end{center}
    
\qs{1.6.a}{The intersection of any $3$ sets is open.}
\sol Let $A$, $B$, and $C$ be open sets. By the definition of a topology, $A \cup B$ is open. Name this set $D$. By the definition of topology, $D \cup C$ is also an open set, therefore the intersection of $3$ open sets is also an open set. 

\qs{1.6.b}{The intersection of finitely many sets is open.}
\sol Consider $x$ sets, $X_a$, $X_b$, $\dots$, $X_z$ such that each of these sets are in $\tau$. Using the answer to the question above, the intersection of these sets will be an open set, until you reach the $x^{th}$ set. 

\qs{1.6.c}{The intersections of open sets is open.}
\sol Consider infinite sets. The intersection of infinite open sets does not neccesarily have to be open. If $X$ is $\mathbb{R}$, then an open set in $\tau$ is an interval, $(a, b)$. If the intersection of these infinite amount of open sets is a single point, ${c}$, then this would not be considered an open set. 

\section{Exercise 1.8}
\qs{1.8}{Describe geometrically what a ball is in $\mathbb{R}$, $\mathbb{R}^2$, and $\mathbb{R}^3$.}
\sol In $\mathbb{R}$, a ball is an open interval. In $\mathbb{R}^2$, a ball is a circle of radius $y$ that excludes all the points lying on the outside of the circle. In $\mathbb{R}^3$, a ball is a sphere that does not include the outermost layer of points on the sphere. 

\section{Exercise 1.9}
\dfn{}{Let $x \in \mathbb{R}^N$ and let $\epsilon > 0$. The \textit{ball} centered at $x$ with radius $\epsilon$ is
    \[
        B_{\epsilon}(x) := {y \in \mathbb{R}^N \mid d(x, y) < \epsilon},
    \]
    where $d(x, y)$ is the Euclidean distance between the points $x$ and $y$. 
}


\qs{1.10}{
    \begin{center}
        Prove that the topology in Definition 1.9 is actually a topology.
    \end{center}
}
\sol To prove this, we must show that the standard topology satisfied the $3$ axioms of the definition of a topology. First, the empty set exists in this topology. If there are no $x$ in $\varphi$, then there is no condition that can checked and $\varphi$ is in $\tau$. An open set $V$ is defined to be a proper subset or subset of $\mathbb{R}^N$, therefore the total set can be included in $\tau$. Now, we must prove that the intersection and union of two open sets is also an open set. Consider the following two lines. 


INSERT DRAWING HERE


For every element $x$ in an open set $V$, there exists a ball centered at $x$ with radius $\epsilon > 0$. Call the two open sets $U$, and $V$. Define their union as $A$. There exists a set $A$ in $\tau$ such that the ball centered at some $x$ with radius $\epsilon > 0$ contains the endpoints that set $U$ and $V$ touch. Therefore, $U \cup V$ will an open set. This same logic applies to the intersection of two sets, claiming that there exists a set $A$ such that $A$ touches the first point of intersection to the second point of intersection of sets $U$ and $V$. 


\section{Example 11}
\begin{center} 
    Show which one of the following examples are open according to Definition 1.9. 
\end{center}

\qs{1.11.a}{The set ${1}$ in $\mathbb{R}$.}
\sol There exists no ball centered at $1$ with raidus $\epsilon < 0$ such that this ball is a subset of ${1}$, therefore this set is not an open set in the standard topology on $\mathbb{R}$. 

\qs{1.11.b}{The interval $(2, 5)$ in $\mathbb{R}$.}
\sol There exists a ball centered at $3$ with radius $\epsilon = 1.5$, therefore this set is an open set in the standard topology on $\mathbb{R}$.  

\qs{1.11.c}{The ball $B_{\epsilon}(x)$ in $\mathbb{R}^N$ for any $x \in \mathbb{R}^N$ and any $\epsilon > 0$.}
\sol This set is an open set, because for all $x$ in this open set $V$, there exists a ball centered at this $x$ that can have a $\epsilon > 0$. This is because this open set will be an interval from $x -  \epsilon$ to $x + \epsilon$, therefore there will always a ball for points between these endpoints centered at $x$ with radius $\mid x - \epsilon \mid$. 

\qs{1.11.d}{The interval $[0, 1)$ in $\mathbb{R}$.}
\sol There exists no ball that is centered at $x = 1$ for this open set, because if $\epsilon > 0$ then the range of this ball will stretch past $1$, which is outside the interval of the set. There exists no ball $B_{\epsilon > )}(1)$, therefore this set cannot be open.  

\qs{1.11.e}{The set $\{(x, y) \in \mathbb{R}^2 \mid x > y\} \text{ in } \mathbb{R}^2$.}
\sol This is not an open set in $\mathbb{R}^2$ because visually, a ball in $\mathbb{R}^2$ visually is a set containing all points in a circle, except those lying on the circumference of the circle. If the $x$ value is greater than the $y$ value for all $(x, y)$ in the circle, then it is not possible for there to be a ball centered at the point with the utmost $x$ value, becuase this point will house a ball (circle) that extends outside of the open set's range. 

\section{Exercise 1.12}

\qs{1.12}{Find all the topologies on the set $X = \{0, 1, 2\}$.}
\sol
\begin{itemize}
  \item $(X, \tau) = \{\phi, X\}$
  \item $(X, \tau) = \{\phi, X, \{0\}\}$
  \item $(X, \tau) = \{\phi, X, \{0, 1\}\}$
  \item $(X, \tau) = \{\phi, X, \{0, 2\}\}$
  \item $(X, \tau) = \{\phi, X, \{1, 2\}\}$
  \item $(X, \tau) = \{\phi, X, \{0, 1, 2\}\}$
  \item $(X, \tau) = \{\phi, X, \{0\}, \{1, 2\}\}$
  \item $(X, \tau) = \{\phi, X, \{0\}, \{0, 1\}\}$
  \item $(X, \tau) = \{\phi, X, \{0\}, \{0, 2\}\}$
  \item $(X, \tau) = \{\phi, X, \{0\}, \{1, 2\}\}$
  \item $(X, \tau) = \{\phi, X, \{0, 1\}, \{1, 2\}\}$
  \item $(X, \tau) = \{\phi, X, \{0, 1\}\}$
  \item $(X, \tau) = \{\phi, X, \{0, 1\}, \{0, 2\}\}$
  \item $(X, \tau) = \{\phi, X, \{0, 2\}, \{1, 2\}\}$
  \item $(X, \tau) = \{\phi, X, \{0, 2\}\}$
  \item $(X, \tau) = \{\phi, X, \{1, 2\}\}$
  \item $(X, \tau) = \{\phi, X, \{0\}, \{1, 2\}\}$
  \item $(X, \tau) = \{\phi, X, \{0, 1\}, \{0, 2\}, \{1, 2\}\}$
\end{itemize}


\section{Fuzzy 1.13}

\textbf{Fuzzy 1.13} Look back at your answer to Exercise $1.12$. Some of those topologies are very similar. One could even say that they are practically "the same topology" with different names. Come up with a definition of what \textit{practically the same topology} could mean. Also, come up with a better name. With this definition, how many essentially different topologies are there on $\{0, 1, 2\}$?



\sol Each topology on $X$ contains $X$ by definition. Excluding this property, each topology contains from $0$ to $7$ different subsets of $X$, and we call classify the topologies of $X$ based on the amount of subsets existing in $\tau$ excluding $X$ and $\phi$. We can call this topology an \textit{n-topology}, where $n$ represents the amount of subsets in each topology of $X$. 



         This would mean there are 
\begin{center}
    \begin{itemize}
         \item 3 1-topologies
         \item 11 2-topologies
         \item 1 3-topology
    \end{itemize}
\end{center}

\end{document}


