\documentclass{report}

\input{preamble}
\input{macros}
\input{letterfonts}
\title{\Huge{Topology Packet}\\Alfonso Gracia - Saz}
\author{\huge{Saahil Sharma}}
\date{}
\begin{document}

\maketitle
\newpage
\pdfbookmark[section]{\contentsname}{toc}
\tableofcontents
\pagebreak

\chapter{}
\section{Exercise 1.2}
\begin{center} 
Among the following, some are topologies on the set $\mathbb{Z}$ and some are not. Which ones are? If an example is not a topology, but you can modify it slightly to make it into a topology, do so. If an example is a topology, and you can generalize it into more examples, do so.  
\end{center}
\qs{Exercise 1.2.a}{\[\tau = \{V \subseteq \mathbb{Z} \mid 0 \in V \} \text{ In other words, a set is open iff it contains } 0.\} \]}
\sol This is not a topology on $\mathbb{Z}$ because all sets must contain the element $0$, therefore the empty set will not be included and $\tau$ is not a valid topology. 

\qs{Exercise 1.2.b}{\[\tau = \{V \subseteq \mathbb{Z} \mid 0 \notin V \} \text{ In other words, a set is open iff it does not contain } 0.\} \]}
\sol This is not a valid topology on $X$, because if $0$ is not included, the total set will not be an open set, therefore $\tau$ is not valid.  

\qs{Exercise 1.2.c}{\[\tau = \{V \subseteq \mathbb{Z} \mid 0 \in V \} \text{ or } 1 \in V\} \]}.
\sol This is a topology on $\mathbb{Z}$ because it contains the empty set and the total set, and a union or intersection of $2$ open sets will be also be considered an open set. 

\qs{Exercise 1.2.d}{\[\tau = \{V \subseteq \mathbb{Z} \mid 0 \in V \} \text{ and } 1 \in V\} \]}.
\sol This is a topology on $\mathbb{Z}$ because the empty set will be included because the elements $0$ and $1$ do not have to be in every open set.

\qs{Exercise 1.2.e}{\[\tau = \{V \subseteq \mathbb{Z} \mid V \text{ is finite }\]}
\sol This is not a topology on $\mathbb{Z}$ as it requires that all sets inside must be finite and the total set is infinite.                                                         

\qs{Exercise 1.2.f}{\[\tau = \{V \subseteq \mathbb{Z} \mid V \text{ is infinite }\]}
\sol The total set is an infinite set, therefore this topology is invalid. 

\section{Exercise 1.3}
\begin{center}
    Among the following, which ones are topologies on the set $\mathbb{R}$ and which ones are not?
\end{center}
\qs{1.3.a}{\[\tau = \{(a, \infty) \mid a \in \mathbb{R}\} \cup \{\phi, \mathbb{R}\}\]}
\sol We can prove this topology is not valid by a proof of contradiction. Consider the claim to be true. Then, the union of $2$ open sets must also be an open set. Consider the $2$ sets where the first set has a starting value of $a$ and the second set has a starting value of $b$. Consider $a < b$. In this situation, the union of these $2$ sets will contain all values from $a$ to $\infty$, not inclusive. This set will not include $b$ though, therefore there is a missing value between $a$ and $\infty$ and the union of these two sets cannot be an open sets. This satisfies that (a) cannot be a valid topology. 

\qs{1.3.b}{\[\tau = \{[a, \infty) \mid a \in \mathbb{R}\} \cup \{\phi, \mathbb{R}\}\].} 
\sol We can prove this topology is valid by satisfying all $3$ axioms that define a topological space. The first axiom is automatically satisfied as the set is in union with the total set and the empty set. Continuing on, the union of $2$ sets must also be an empty set. This can be proved by considering $2$ sets, one with an $a$ value of $a$ and another with an $a$ value of $b$. If $a = b$, then their union is itself and the resulting set is open. Continuing on, if $a < b$, then the set will include all values from $a$ to $\infty$, including $b$, therefore this set would also be an open set. The same logic can be applied to the situation in which $b < a$, and for the $3$rd axiom, regarding intersections of $2$ sets. 

\section{Exercise 1.4}
\begin{center}
Let $X$ be any set. 
\end{center}
\qs{1.4.a}{What is the topology on $X$ that has the most open sets? This is called the \textit{discrete} topology on $X$.} 
\sol The topology with the most amount of open sets on $X$ will include the total set, the empty set, and every subset, and the power set $\mathbb{X}$ that includes all the subsets of $X$. 
\qs{1.4.b}{What is the topology on $X$ that has the least open sets? This is called the \textit{indiscrete} topology on $X$.}
\sol The topology with the least amount of open sets would only contain the total set, the empty set, and the union of those $2$ sets.  
\newpage
\section{Exercise 1.5}
\begin{center}
Let $X$ be an arbitrary set. Which ones of the following are topologies?
\end{center}

\qs{1.5.a}{The \textit{cofinite} topology: A set \[V \subseteq X \text{ is open iff } [ X \backslash  V \text{ is finite or } V = \phi]\].}
\sol We must prove this topology $\tau$ is in order with the $3$ defining points of a topology on $X$. 
Now, if we have two sets in $\tau$ being $V_i$ and $V_j$, then $X - (V_i \cup V_j)$ must also be infinite. This is from the second axiom of a topology. From De Morgan's Laws, we know that
                    \[
                        X - ( V_i \cup V_j ) = (X - V_i) \cap (X - V_j).                  
                    \]
This indicates the intersection of $2$ infinite sets, which can be either infinite, finite or empty. Therefore, there exists a counterexample and the \textit{coinfinite} topology does not exist for all sets $X$. 

\qs{1.5.c}{The \textit{cocountable} topology: A set $V \subseteq X$ is open iff \[ [X \backslash V \text{ is countable or  } V = \phi]\].} 
\sol To prove this, we must prove that the topology $\tau$ exists only for countable sets. That is, we have already proved when $X$ is finite in the first case. Let $K_i$ be equal to $X \backslash V_i$. Therefore, $\tau$ is equal to the collection of sets $V_i$, $V_j$, $\dots$. Therefore, we must prove that the union of $V_i$ and $V_j$ is also an open set. To do this, we use De Morgan's laws.
                    \[X - (V_i \cup V_j) = (X-V_i) \cap (X-V_j) \Longrightarrow  \]
                    \[X - (V_i \cup V_j) = (K_i) \cap (K_j) \].
By definition, both $K_i$ and $K_j$ are countable, therefore their union is countable. The same logic and be applied to the $3$rd axiom of a topology, proving that $\tau$ is a valid topology. A very nice question!                            

\section{Exercise 1.6}
\begin{center}
    Let $(X, \tau)$ be a topological space. Prove each of the following statements true or false.         
\end{center}
    
\qs{1.6.a}{The intersection of any $3$ sets is open.}
\sol Let $A$, $B$, and $C$ be open sets. By the definition of a topology, $A \cup B$ is open. Name this set $D$. By the definition of topology, $D \cup C$ is also an open set, therefore the intersection of $3$ open sets is also an open set. 

\qs{1.6.b}{The intersection of finitely many sets is open.}
\sol Consider $x$ sets, $X_a$, $X_b$, $\dots$, $X_z$ such that each of these sets are in $\tau$. Using the answer to the question above, the intersection of these sets will be an open set, until you reach the $x^{th}$ set. 

\qs{1.6.c}{The intersections of open sets is open.}
\sol Consider infinite sets. The intersection of infinite open sets does not neccesarily have to be open. If $X$ is $\mathbb{R}$, then an open set in $\tau$ is an interval, $(a, b)$. If the intersection of these infinite amount of open sets is a single point, ${c}$, then this would not be considered an open set. 


\end{document}


