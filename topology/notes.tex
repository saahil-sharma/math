\documentclass[a4paper]{report}
\title{Notes for Topology: Munkres}
\author{Saahil Sharma}
\usepackage{amssymb}
\usepackage{titling}
\usepackage{amsmath}
\usepackage{amsfonts}
\usepackage[english]{babel}
\newtheorem{theorem}{Theorem}
\begin{document}
\begin{titlingpage}
\maketitle
\begin{abstract}
	Welcome! This document will be used to create notes on Topology for the first 3 chapters of Munkres. Problems and exercises will be completed in a different document. All notes will be written by Saahil Sharma. Text Editor being used is Nvim with WSL 2022. Notes are first dated at 12/12/2023. Thank you.  
\end{abstract}
\end{titlingpage}
\section{Chapter 1}

\begin{itemize}

\item Notation

	If an object $\textit{a}$ belongs to a set $\textit{A}$, we denote this by writing 
		$$\textit{a}\in \textit{A}$$.
	If the converse is true - The element $\textit{a}$ does not belong to a set $\textit{A}$, we denote this by writing 
		$$\textit{a}\notin \textit{A}$$.
	We say that $\textit{A}$ is a subset of $\textit{B}$ if all of the elements of $\textit{A}$ are also in $\textit{B}$, denoted by writing
		$$\textit{A}\subset\textit{B}$$.
	If $\textit{A}$ is not the same as $\textit{B}$, we call  $\textit{A}$ a proper subset of $\textit{B}$, denoted with 
		$$\textit{A}\subseteq\textit{B}$$.
	These relations are called inclusion and proper inclusion, respectively. It can be read as "$B$ contains $A$". 
 
	How can one specify a set?

	If a set is short, one can simply list the elements of the set:
		$$\textit{A} = \{a, b, c \}. $$
\item Set Builder Notation:

If a set is too long, or possibly infintely long, we can use set builder notation to list the items within the set. To define the set $B$ using set builder notation, we denote $B$ as 
		$$ \textit{B} = \{ x \mid \text{x is an even integer }\}. $$ 
In this example, $B$ would be the set of all even integers. 

The union of two sets is defined as 
		$$ \textit{A}\cup\textit{B} = \{x \mid x \in A \text{or} x \in B\}. $$
The intersection of two sets in defined as 
		$$ \textit{A}\cap\textit{B} = \{x \mid x \in A \text{and} x \in B\}. $$
What if $A$ and $B$ have no elements in common? To denote this set, we use the empty set, which is written as 
		$$ \varnothing = \{ \}. $$
This fact can also be expressed by saying $A$ and $B$ are \textit{\textbf{disjoint}}.
		
\item 
	The Difference of 2 Sets

	Think about this as litterally subtracting out the terms that exist in $A$ and $B$. To define this with rigor, we can write
			$$A - B = \{\, x \mid x \in A \text{ and } x \notin B \,\}.$$
	This can sometimes be called the \textit{\textbf{complement}} of $B$ relative to $A$.
\item Rules of Set Theory
	\begin{itemize}
		\item 1
			$$A\cap \left( B \cup C \right) = \left( A \cap B \right) \cup \left( A \cap C \right) $$
			This sort of resembles the distributive property. 
		\item 2 
			$$A\cup \left( B \cap C \right) = \left( A \cup B \right) \cap \left( A \cup C \right) $$
			This also shares a resemblence to the distrubitive property. 
		\item 3	$$A - \left( B \cup C \right) = \left( A - B \right) \cap \left( A - C \right) $$
 
		\item 4 $$A - \left( B \cap C \right) = \left( A - B \right) \cup \left( A - C \right) $$
			An efficient way to memorize this:
			\begin{itemize}
				\item 1 \textit{The complement of the union equals the intersection of the complements.}
				\item 2 \textit{The complement of the intersection equals the union of the complements.} 
			\end{itemize}
	\end{itemize}

\item Sets
	
	A set can consist of anything, from other numbers to sets themselves. You can have a set of of all the subsets in a set:
			$$ A = \{\, A \mid A \text{ is a subset of } B \,\}.$$
		This is sometimes called the power set of $B$, denoted as $\mathcal{B}$.
	A set whose elements are set is commonly referred to as a collection of sets. 
	
		We now make a distinction in notation. To illustrate, if $A$ is the set $\{1, 2, 3\}$, then the following are all equivalent. 
		$$a \in A,     \{a\} \subset A, \text{and} \{a\} \in  \mathcal{P}(A). $$
	Given a collection of sets $\mathcal{A}$, the \textbf{union} of $\mathcal{A}$ is defined by the equation 
		 \[
			 \bigcup_{x \in \mathcal{A}} A = \{\, x \mid x \in A\text{ for at least one } A \in \mathcal{A} \,\}.
		 \]
	The intersection of the elements a is defined by the equation 
		\[
			 \bigcap_{x \in \mathcal{A}} A = \{\, x \mid x \in A\text{ for every } A \in \mathcal{A} \,\}.
		\]
	This intersection is not defined when $\mathcal{A}$ is empty.

\item Cartesian Products
	Given sets $A$ and $B$, we can define their cartesian product to be 
	\[
		A \times B = \{ (a, b) \mid a \in A  \text{ and }  b \in B \}
	\]
	The cartesian product is the set of all ordered pairs, $(a, b)$, such that $a$ is an element of $A$ and $b$ is an element of $B$.

\item Rule of Assignment

	A Rule of Assignment $r$ is of $C \times D$ is a rule of assignment if 
		\[
			[(c, d) \in r \text{ and } (c, d') \in r] \Longrightarrow [d = d'].
		\]
	We think of $r$ as a way of assigning, to the element $c$ of $C$, the element $d$ of $D$, for which (c, d) $\in$ $r$. 

		Given a rule of assignment $r$, the \textit{\textbf{domain}} of $r$ is defined to be the subset of $C$ consisting of all the first coordinates of elements of $r$, and the \textbf{image set} of $r$ is defined as the subset of $D$ consisting of all second coordinates of elements of $r$. Formally, 
		\[
			\text{domain } r = \{c \mid \text{ there exists } d \in D \text{ such that } (c, d) \in r \}, 
			 \text{image } r = \{d \mid \text{ there exists } c \in C \text{ such that } (c, d) \in r \}.
		\]
\item Function
	A function $f$is a rule of assignment $r$, together with a set $B$, that contains the image set of $r$. The domain $A$ of the rule $r$ is also called the domain of the function $f$; the image set of $r$ is also called the \textbf{image set} of $f$; and the set $B$ is called the \textbf{range} of $f$. 

	If $f$ is a function having domain $A$ and range $B$, we express this fact by writing 
	\[
		f: A \Longrightarrow B,
	\]
	which is read "$f$ is a function from $A$ to $B$, or "$f$ is a mapping from $A$ into $B$," or simply "$f$ maps $A$ into $B$." One sometimes visualizes $f$ as a geometric transformation phsyically carrying the points of $A$ to points of $B$. 

	If $f: A \Longrightarrow B$ and if $a$ is an element of $A$, we denote by $f(a)$ the unique element of $B$ that the rule determining $f$ assigns to $a$; it is called the $textbf{value}$ of $f$ at $a$, or sometimes the \textbf{image} of $a$ under $f$. Formally, if $r$ is the rule of the function $f$, then $f(a)$ denotes the unique element of $B$ such that $(a, f(a)) \in r$. 
		Using this notation, one can go back to defining functions almost as one did before, with no lack of rigor. For instance, one can write (letting $\mathbb{R}$ denote the real numbers)
		\[
			\text{"Let } f \text{be the function whose rule is } \{(x, x^3 + 1) \mid x \in \mathbb{R} \text{ and whose range is } \mathbb{R}\text{",}
		\]
	or one can equally write
		\[ 
			\text{"Let } f : \mathbb{R} \Rightarrow \mathbb{R} \text{be the function such that } f(x) = x^3 + 1\text{."}
		\]
	The sentence let $f$ be the function $f(x) = x^3 +1$ is no longer adequate for specifying a function becuase it neither specifies the domain or range of $f$. 
	
\item Restriction
		IF $f : A \Rightarrow B$ and if $A_{0}$ is a subset of A, we define the restriction of $f$ to $A_{0}$ to be the function mapping $A_{0}$ into $B$ whose rule is 
		\[
			\{(a, f(a)) \mid a \in A_{0}\}. 
		\]
	It is denoted by $f/mid A_{0}$, which is read $"f \text{ restricted to } A_{0}$. 
	
\item Composite Functions
		Given functions $f : A \longrightarrow B$ and $g: B \longrightarrow C$, we define the \textbf{composite} $g \circ f$ of $f$ and $g$ as the function $g \circ f : A \longrightarrow C $ defined by the equation $(g \circ f)(a) = g(f(a))$. 
		Formally, $g \circ f : A \longrightarrow C$ is the function whose rule is
		\[
			\{(a, c) \mid \text{For some } b \in B, f(a) = b and g(b) = c\}. 
		\]
		In your mind, you can picture this as the point $a$ moving to the point $f(a)$, and then to the point $g(f(a))$. 
		Note that $g \circ f$ is only defined when the range of $f$ \textit{equals} the domain of $g$. 

\item Bijective, Injective, and Surjective
	A function $f: A \longrightarrow B$ is said to be \textbf{injective} (or one to one) if for each pair of distinct points of $A$, their images under $f$ are distinct. It is said to be \textbf{surjective} (or $f$ is said to map $ A \textbf{onto} B$) if every element of $B$ is the image of some element of $A$ under the function $f$. IF $f$ is both injective and surjective, it is said to be \textbf{bijective} (or is called a \textbf{one to one correspondence}). 
		More formally, $f$ is injective if
		$$
			[f(a) = f(a')] \Longrightarrow [a = a'], 
		$$ 
		and $f$ is surjective if 
		$$ 
			[b \in B] \Longrightarrow [b = f(a) \text{ for at least one } a \in A]. 
		$$ 
	Injectivity of $f$ depends only on the rule of $f$; surjectivity depends on the range of $f$ as well. You can check that the composite of two injective functions is injective, and the composite of two sujective functions is surjective; it follows that the composite of two bijective functions is bijective. 
	
	If $f$ is bijective, there exists a function from $B$ to $A$ called the \textbf{inverse} of $f$. IT is denoted by $f^-1$, and is the standard inverse of $f$ as you have learned. IF $f$ is surjective, this implies there exists such an element in the domain that yeilds an inverse relationship, and if $f$ is injective then there is \textit{only one} such element that exists. 

	Lemma 2.1. Let $f: A \longrightarrow B$. If there are functions $g: B \longrightarrow A$ and $h: B \longrightarrow A$ such that $g(f(a)) = a$ for every $a$ in $A$ and $f(h(b)) = b$ for every $b$ in $B$, then $f$ is bijective and $g = h = f^-{1}$. 

\item Image
	Let $f: A \longrightarrow B$. If $A_{0}$ is a subset of $A$, we denote by $f(A_{0})$ the set of all images of points of $A_{0}$ under the function $f$; this set is called the \textbf{image} of $A_{0}$ under $f$. Formally, 
	\[
		f(A_{0}) = \{ b \mid b = f(a) \text{ for at least one } a \in A_{0}\}. 
	\]

	On the other hand, if $B_{0}$ is a subset of $B$, we denote by $f^{-1}(B_{0})$ the set of all elements of $A$ whose images under $f$ lie in $B_{0}$; it is called the \textbf{preimage} of $B_{0}$ under $f$, also called the "counterimage" or the "inverse image". Formally, 
	\[
		f^{-1}(B_{0}) = \{a \mid f(a) \in B_{0}\}. 
	\]

	
\end{itemize}

	



\end{document}
