\documentclass[9pt]{beamer}
\usetheme{Frankfurt}
\usepackage{asymptote}
\usepackage{tikz}
\usepackage{amsmath}
\usepackage{mathtools}
\usepackage{fontawesome5}
\usepackage{xcolor}
\usepackage[outline]{contour}
\newcommand*{\fullchili}{\textcolor{red}{\faPepperHot}}
\newcommand*{\emptychili}{\textcolor{white}{\contour{red}{\faPepperHot}}}
\usepackage{setspace}
\setstretch{1.15}

\makeatletter
\renewrobustcmd{\beamer@@pause}[1][]{%
  \unless\ifmeasuring@%
  \ifblank{#1}%
    {\stepcounter{beamerpauses}}%
    {\setcounter{beamerpauses}{#1}}%
  \onslide<\value{beamerpauses}->\relax%
  \fi%
}
\makeatother

\begin{document}

\begin{frame}
\frametitle{What is a set? ~ \emptychili \emptychili \emptychili \emptychili}
    \bigskip
    A \textbf{set} $X$ is a collection of items. 
    A set $X$ can be used to define a group of numbers, such as the set $\mathbb{R}$ of all real numbers. 
    
    Let's look at some examples of a \textbf{set}. 

    What do the following sets represent? 
    \begin{itemize} 
        \item $\mathbb{Q}$

       \item $\mathbb{Z}$

        \item $\mathbb{N}$
    \end{itemize}
\end{frame}
    
\begin{frame}
\frametitle{Set Examples  ~ \emptychili \emptychili \emptychili \emptychili}
    \bigskip
    A \textbf{set} $X$ is a collection of items. \bigskip
    A set $X$ can be used to define a group of numbers, such as the set $\mathbb{R}$ of all real numbers. 
    \bigskip
    Lets look at some examples of a \textbf{set}. 

    What do the following sets represent? 
    \begin{itemize} 
        \item $\mathbb{Q}$: The set of rational numbers. 
       
        \item $\mathbb{Z}$: 

        \item $\mathbb{N}$:
    \end{itemize}
\end{frame}

\begin{frame}
\frametitle{Set Examples  ~ \emptychili \emptychili \emptychili \emptychili}
    \bigskip
    A \textbf{set} $X$ is a collection of items. \bigskip
    A set $X$ can be used to define a group of numbers, such as the set $\mathbb{R}$ of all real numbers. 
    \bigskip
    Lets look at some examples of a \textbf{set}. 

    What do the following sets represent? 
    \begin{itemize} 
        \item $\mathbb{Q}$: The set of rational numbers. 
       
        \item $\mathbb{Z}$: The set of all integers. 

        \item $\mathbb{N}$:
    \end{itemize}
\end{frame}

\begin{frame}
\frametitle{Set Examples  ~ \emptychili \emptychili \emptychili \emptychili}
    \bigskip
    A \textbf{set} $X$ is a collection of items. \bigskip
    A set $X$ can be used to define a group of numbers, such as the set $\mathbb{R}$ of all real numbers. 
    \bigskip
    Lets look at some examples of a \textbf{set}. 

    What do the following sets represent? 
    \begin{itemize} 
        \item $\mathbb{Q}$: The set of rational numbers. 
       
        \item $\mathbb{Z}$: The set of all integers. 

        \item $\mathbb{N}$: The set of all natural numbers. 
    \end{itemize}
\end{frame}

\begin{frame}
\frametitle{Subsets ~ \emptychili \emptychili \emptychili \emptychili}
    \bigskip
    A \textbf{subset} $V$ is a set such that all elements are contained in another set. \bigskip
    \medskip 
    A subset $C$ of $V$ can be written as $C \subset V$. This can read as $V$ contains $C$.

    What are some example subsets of the following sets?
    \medskip
        \begin{itemize} 
        \item $\mathbb{R}$: The set of all real numbers. 

        \item $\mathbb{Q}$: The set of rational numbers. 
       
        \item $\mathbb{Z}$: The set of all integers. 

        \item $\mathbb{N}$: The set of all natural numbers.
    \end{itemize}
\end{frame}

\begin{frame}
    \frametitle{Topology  ~ \emptychili \emptychili \emptychili \emptychili}
    \bigskip
    A \textbf{subset} $V$ is a set such that all elements are contained in another set. \bigskip

    \bigskip 
    A subset $C$ of $V$ can be written as $C \subset V$. This can read as $V$ contains $C$.

    What are some example subsets of the following sets?
    \bigskip
    \begin{itemize} 
        \item $\mathbb{R}$: The set of all real numbers. 

        \item $\mathbb{Q}$: The set of rational numbers. 

        \item $\mathbb{Z}$: The set of all integers. 

        \item $\mathbb{N}$: The set of all natural numbers.
    \end{itemize}
\end{frame}

\begin{frame}
\frametitle{Set Builder Notation ~ \fullchili \fullchili \emptychili \emptychili}
    Set builder notation is a notation that can be used to describe sets neatly and quickly. 
    Set builder notation generally follows the following conventions. 
    \[
        V = \{\text{ element } \mid \text{ condition } \} 
    \] 
    Some textbooks may use a colon $:$ instead of the bar, $\mid$. Essentially, this reads the set $V$ contains all elements such that the condition is satisfied. Set builder notation is very useful for describing sets such that all terms satisfy a condition or pattern. 
   
    Let's do some examples. 
\end{frame}
\begin{frame}
    \frametitle{Set Builder Notation Examples ~ \fullchili \fullchili \fullchili \emptychili}
    
    Turn and talk to a partner about the following sets. 

    What do these sets mean?

    \begin{itemize}
        \item $X = \{x \mid x = 2n - 1 \}$, $n \in \mathbb{N}$ 

        \item $X = \{x \mid x = 2n \}$, $n \in \mathbb{N}$ 

        Consider $X$ to be a arbitrary set. Define $V$ to be a subset of $X$.  
        
    \item $ V = $
    \end{itemize}
    
\end{frame}



\begin{frame}
    \frametitle{Topology ~ \emptychili \emptychili \emptychili \emptychili}
    A topology $\tau$ on a set $X$ is a group of subsets of $X$ that satisfy $3$ conditions. A subset that satisfies these conditions is called \alert{\textit{open}}. Consider the sets $U$ and $V$. 
    \begin{itemize}
        \item The total set $X$ and the empty set $\emptyset$ must be open sets. 
            
        \item The union of two open sets must be an open set. For example, $U \cup V$ must be an open set. 

        \item The intersection of two open sets must be an open set. For example, $U \cap V$ must be an open set. 
    \end{itemize}
    
    A topological space is a pair $(X, \tau)$ where $X$ is a set and $\tau$ is a topology on $X$.  
\end{frame}

\begin{frame}
    \frametitle{Some Exercises ~ \emptychili \emptychili \emptychili \emptychili}
    Let $(X, \tau)$ be a topological space. Prove each of the following statements \textit{true} or \textit{false}.  
    \begin{itemize}
       \item The intersection of any $3$ open sets is an open set. 

        \item The intersection of finitely many open sets is open.

        \item The intersection of open sets is open. 
    \end{itemize}
\end{frame}

\end{document}






